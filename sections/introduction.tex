\section{Introduction}

Reinforcement Learning (RL) is a machine learning paradigm in which an agent learns to make decisions by interacting with an environment and receiving feedback in the form of rewards. Through trial-and-error, the agent improves its policy to maximize cumulative rewards. In recent years, deep RL has solved complex tasks in sumulation - from playing video games to controlling simulated robots - showcasing the protential of automomous learning. However, applying these learned policies on pysical systems presents a significant challenge known as the Sim-to-Real transfer or reality gap. Simulated training is appealing because collecting real-world data can be costly, slow, or unsafe, whereas simulations provide virtually unlimited, fast data in a safe setting. Yet differences between simulated and real physics, sensors, and visuals often cause a policy that works well in simulation to fail on the real hardware.Bridging this sim-to-real gap has become a crucial research focus in fileds like robotics and atuonomous vehicles.

To address the reality gap, researchers have developed a spectrum of transfer techniques. At one end, one can improve the simulators's fidelity through \textit{system identification} - calibrating simulation parameters to better match reality. At the other end, one can make the learned policy robust to simulation imperfections via \textit{domain randomization} - training on a wide variety of simulated conditions so that the real world appears as just another variation. In between, \textit{domain adaptation} methods seek to transform or adjust simulated data (or learned models) to resembele real-world data, for example by using generative models to make simulated camera images look realistic. Each approach has advantages and limitations, and often a combination is needed for succcess. Indeed, closing the sim-to-real gap is not only about transferring a policy once, but also about ensuring the policy remains reliable under real-world variability und unforeeseen conditions. 

This paper provides a detailed overview of sim-to-real transfer in reinforcement learning, with an emphasis on clarity and cross-domain coverage. \textbf{Our contributions are as follows}: (1) We introduce core concepts of RL and simulation, defining the reality gap and why it arises. (2) We describe key sim-to-real transfer techniques - domain randomization, domain adaptation, and system identification - and illustrate how they work. (3) We survey case studies  of sim-to-real success in diverse  domains: robotic object manipulation, drone navigation, and legged robot locomotion. (4) We discuss current challenges (e.g. simulator fidelity, safety, generalization) and future directions such as combining methods and improving sample efficiency. 

The remainder of this paper is organized as follows. \textbf{Section II} provides background on reinforcement learning and simulation environments, and defines the sim-to-real gap. \textbf{Section III} presents the main sim-to-real transfer techniques (domain randomization, domain adaptation, system identification). \textbf{Section IV} showcases application case studies across different domains. \textbf{Section V} discusses open challenges and future research directions. Section VI concludes this paper.