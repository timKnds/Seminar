%%%%%%%%%%%%%%%%%%%%%%%%%%%%%%%%%%%%%%%%%
% Journal Article
% LaTeX Template
% Version 1.4 (15/5/16)
%
% This template has been downloaded from:
% http://www.LaTeXTemplates.com
%
% Original author:
% Frits Wenneker (http://www.howtotex.com) with extensive modifications by
% Vel (vel@LaTeXTemplates.com)
%
% License:
% CC BY-NC-SA 3.0 (http://creativecommons.org/licenses/by-nc-sa/3.0/)
%
%%%%%%%%%%%%%%%%%%%%%%%%%%%%%%%%%%%%%%%%%

%----------------------------------------------------------------------------------------
%	PACKAGES AND OTHER DOCUMENT CONFIGURATIONS
%----------------------------------------------------------------------------------------

\documentclass[twoside,twocolumn]{article}

\usepackage{blindtext} % Package to generate dummy text throughout this template 

\usepackage[sc]{mathpazo} % Use the Palatino font
\usepackage[T1]{fontenc} % Use 8-bit encoding that has 256 glyphs
\linespread{1.05} % Line spacing - Palatino needs more space between lines
\usepackage{microtype} % Slightly tweak font spacing for aesthetics

\usepackage[english]{babel} % Language hyphenation and typographical rules

\usepackage[hmarginratio=1:1,top=32mm,columnsep=20pt]{geometry} % Document margins
\usepackage[hang, small,labelfont=bf,up,textfont=it,up]{caption} % Custom captions under/above floats in tables or figures
\usepackage{booktabs} % Horizontal rules in tables

\usepackage{lettrine} % The lettrine is the first enlarged letter at the beginning of the text

\usepackage{enumitem} % Customized lists
\setlist[itemize]{noitemsep} % Make itemize lists more compact

\usepackage{abstract} % Allows abstract customization
\renewcommand{\abstractnamefont}{\normalfont\bfseries} % Set the "Abstract" text to bold
\renewcommand{\abstracttextfont}{\normalfont\small\itshape} % Set the abstract itself to small italic text

\usepackage{titlesec} % Allows customization of titles
\renewcommand\thesection{\Roman{section}} % Roman numerals for the sections
\renewcommand\thesubsection{\roman{subsection}} % roman numerals for subsections
\titleformat{\section}[block]{\large\scshape\centering}{\thesection.}{1em}{} % Change the look of the section titles
\titleformat{\subsection}[block]{\large}{\thesubsection.}{1em}{} % Change the look of the section titles

\usepackage{fancyhdr} % Headers and footers
\pagestyle{fancy} % All pages have headers and footers
\fancyhead{} % Blank out the default header
\fancyfoot{} % Blank out the default footer
\fancyhead[C]{Seminar Data Science \& Artificial Intelligence $\bullet$ SS25 proceedings} % Custom header text
\fancyfoot[RO,LE]{\thepage} % Custom footer text

\usepackage{titling} % Customizing the title section

\usepackage{hyperref} % For hyperlinks in the PDF

\usepackage{lisitings}

%----------------------------------------------------------------------------------------
%	TITLE SECTION
%----------------------------------------------------------------------------------------

\setlength{\droptitle}{-4\baselineskip} % Move the title up

\pretitle{\begin{center}\Huge\bfseries} % Article title formatting
\posttitle{\end{center}} % Article title closing formatting
\title{Sim-to-Real in Reinforcement Learning Across Multiple Domains} % Article title
\author{%
\textsc{Tim Knudsen} \\[1ex] % Your name
\normalsize FH Wedel University of Applied Sciences\\ % Your institution
\normalsize \href{tim04knudsen@gmail.com}{tim04knudsen@gmail.com} % Your email address
}
\date{\today} % Leave empty to omit a date
\renewcommand{\maketitlehookd}{%
\begin{abstract}
\noindent Reinforcement Learning is a modern and powerful approach to machine learning. It has achived remarkable success in simulated environments. But no matter how powerful your algorithm is, it is only as good as the deployment in the real world. Due to the complexity of the real world, the Sim-2-Real transfer is a major challenge in the field of Reinforcement Learning. In this paper, we will discuss the challenges of the Sim-2-Real transfer and how to overcome them. We will discuss different approaches to the Sim-2-Real transfer and how they can be used to improve the performance of Reinforcement Learning.
\end{abstract}
}

%----------------------------------------------------------------------------------------

\begin{document}

% Print the title
\maketitle
%----------------------------------------------------------------------------------------
\section{Introduction}
Reinforcement Learning (RL) is a machine learning paradigm in which an agent learns to make decisions by interacting with an environment and receiving feedback in the form of rewards. Through trial-and-error, the agent improves its policy to maximize cumulative rewards. In recent years, deep RL has solved complex tasks in sumulation - from playing video games to controlling simulated robots - showcasing the protential of automomous learning. However, applying these learned policies on pysical systems presents a significant challenge known as the sim-to-real gap. Simulated training is appealing because collecting real-world data can be costly, slow, or unsafe, whereas simulations provide virtually unlimited, fast data in a safe setting. Yet differences between simulated and real physics, sensors, and visuals often cause a policy that works well in simulation to fail on the real hardware. Bridging this sim-to-real gap has become a crucial research focus in fields like robotics and atonomous vehicles.To address the reality gap, researchers have developed a spectrum of transfer techniques.

This paper revises the core sections of sim-to-real transfer in RL by organizing the discussion around the fundamental components of an RL problem, following the Observation–Action–Transition–Reward (OATR) framework proposed by recent work. We first introduce RL fundamentals in terms of these four components, highlighting how each contributes to the sim-to-real discrepancy. We then provide a structured review of sim-to-real transfer techniques, explicitly mapping each method to the OATR element it targets. We preserve key references and seminal examples from prior research – including CAD2RL for drone navigation, OpenAI’s Dactyl and ADR (Automatic Domain Randomization) for robotic manipulation, and Tan et al.’s quadruped locomotion – to illustrate each category of method.
\section{Background}
\subsection{Definition}
An RL task is an MDP characterized by observation (state) $s$, action $a$, transition $T(s,a\to s')$, and reward $r$. In simulation, these are often idealized: perfect sensors, discretized actions, deterministic physics, and easily computable rewards. In reality, by contrast, sensors produce noisy observations, actuators have latency and limits, dynamics are stochastic and hard to model, and rewards may be sparse or indirect. Table~\ref{tab:elements} summarizes typical sim-versus-real discrepancies. Any of these mismatches can cause failure: for example, a vision policy may overfit simulated textures and fail on real images, or an action sequence learned without actuator latency may be unsafe on hardware.

\begin{table}[htbp]
    \centering
    \caption{Simulated vs real MDP components: sources of discrepancy (simulated vs real).}
    \begin{tabular}{|p{0.3\linewidth}|p{0.6\linewidth}|}
        \hline
        \textbf{Observation:} & Idealized vs noisy vision/depth; synthetic vs real textures (rendering) \\
        \hline
        \textbf{Action:} & Precise, discrete commands vs actuator noise, latency, continuous control \\
        \hline
        \textbf{Transition:} & Deterministic physics vs stochastic dynamics, friction, contact variations \\
        \hline
        \textbf{Reward:} & Engineered or dense reward vs sparse/uncertain task objectives \\
        \hline
    \end{tabular}
    \label{tab:elements}
\end{table}

Each \simtoreal method typically addresses one or more of these elements. For instance, \emph{domain randomization} perturbs visuals and/or physics during training to cover a wide range of possible observations or transitions {\cite{Tobin2017,Sadeghi2017}}. In contrast, \emph{system identification} methods use real data to calibrate the simulator’s parameters {\cite{Chebotar2019}}. Some approaches combine both: e.g.\ OpenAI’s ADR gradually expands randomization range, effectively automating calibration while training {\cite{Akkaya2019}}. We now detail techniques by OATR component, starting with observations.

% These four elements together determine the behaviour of an RL agent. The \simtoreal challenge can be viewed as discrepancy in one or more of these elements between the simulator MDP and the real-world MDP. Each technique can be used to mitigate one or more type of discrepancy.
\usepackage{listings}
\section{Sim-to-Real Transfer Techniques}

Researchers have developed a variety of techniques to enable policies learned in simulation to work in the real world. Figure 1 illustrates three common strategies: system identification (making the sim more like reality), domain adaptation (making the data or model translate between sim and real), and domain randomization (exposing the policy to diverse simulated experiences so it generalizes to real). We will discuss each in detail, including how they are implemented and examples of their use in RL. It is important to note that these techniques are often complementary – for example, one might first apply system identification to get a reasonably accurate simulator, then use domain randomization during training, and perhaps a form of adaptation at deployment. The optimal approach can depend on the task and what aspects of the sim-to-real gap are most significant (e.g. visual differences vs. physics differences).

\subsection{Domain Randomization}
Concept: Domain randomization (DR) is a strategy where, instead of trying to make the simulator perfectly realistic, we intentionally randomize various aspects of the simulation during training. The idea, introduced by Sadeghi and Levine (2016) and Tobin et al. (2017), is that by exposing the RL agent to a wide range of environment variations, the learned policy will focus on task-relevant features and become robust to changes. If the randomization is sufficiently broad, the real world – with its particular configuration – will appear to the agent as just another variation it has seen in training. In effect, domain randomization enlarges the training domain to encompass the real domain.

What to Randomize: Practically any parameter of the simulation that could differ in reality is a candidate for randomization. Common randomizations include: object colors and textures, lighting conditions, background imagery, positions and shapes of objects, sensor noise, and physics properties like masses, friction coefficients, and joint damping. For example, in a simulated robotic vision
task, one might randomize the colors and textures of walls and floors each episode, add random noise or blur to the camera images, and randomize lighting direction and intensity. In a dynamics-centric task like robot locomotion, one might randomize gravity slightly, vary the robot’s motor torque scalars, or add random forces (perturbations) during the episode to simulate bumps or wind. The randomization
ranges should be chosen to cover the plausible real-world variations – often initially set wide when unsure. Notably, domain randomization does not attempt to perfectly mimic reality; it instead says “throw everything at the agent, and if it can handle all of it, it will handle reality.”

Effect on Learning: During training, each episode or training iteration uses a differently randomized environment. The RL algorithm, say Proximal Policy Optimization (PPO), optimizes the policy to maximize expected reward over this distribution of environments (rather than one fixed environment). Mathematically, if $\xi$ denotes a vector of simulation parameters to randomize (both visual and physical), domain randomization seeks a policy $\pi_\theta$ that performs well under the expectation over $\xi \sim \Xi$, where $\Xi$ is the space of randomization. In other words, the objective
becomes:
$$\max_\theta \; \mathbb{E}{\xi \sim \Xi}\left[\, \mathbb{E}[R\tau]\right],(\xi)$$
where $R(\tau)$ is the cumulative reward of a trajectory $\tau$ sampled from the simulator with randomization $\xi$. By optimizing this, the agent learns to handle all the randomized conditions, rather than overfitting to one setting.

In pseudocode, a domain randomization training loop might look like this:

\begin{verbatim*}
    initialize policy parameters
    for iteration = 1 to N:
        sample random parameters
\end{verbatim}
%----------------------------------------------------------------------------------------

\bibliographystyle{abbrv} % "abbrv" reference style
\bibliography{refs} % Entries are in the refs.bib file

%----------------------------------------------------------------------------------------

\end{document}
